\documentclass[twocolumn, a4paper]{hcresume}

% 日本語処理のためのパッケージ
\usepackage{otf}
\usepackage{hyperref}
\usepackage[dvipdfmx]{pxjahyper}
\usepackage[dvipdfmx]{graphicx}
\usepackage{bm}
\usepackage{amsmath}
\usepackage{txfonts}
\usepackage{color}
\usepackage{BoldGothic4fig}
\usepackage{subcaption}
\usepackage{float}
\usepackage{here}
\usepackage{listings}
\lstset{
  % Font / spacing
  basicstyle=\ttfamily\scriptsize,
  columns=fullflexible,
  % Layout
  numbers=none,
  frame=tb,
  breaklines=true,
  breakindent=0pt,
  % Reduce vertical whitespace around listings
  aboveskip=0.5\baselineskip,
  belowskip=0.5\baselineskip
}
\newfloat{lstfloat}{htbp}{lop}
\floatname{lstfloat}{Listing}
\def\lstfloatautorefname{Listing}


\hcheader{MIプログラム 卒業研究発表会}
\title{\bf 精緻化した受験者プロフィールを用いた \\ 大規模言語モデルに基づく仮想受験者}
\author{2020029 山羽 亨}
\supervisor{指導教員 宇都 雅輝 准教授}

\begin{document}
\maketitle
\pagestyle{empty}
\thispagestyle{empty}
\section{はじめに}

% 近年ニーズが高まっている個別適応的な学習支援やCBT(Computer Based Testing)などに基づく高度なテスト運用においては,難易度や識別力などの特性が既知のテスト問題(以降,項目と呼ぶ)が重要な役割を果たすが,
% 項目特性の推定には,予め目標の母集団から受験者や学習者(以降,受験者に統一)を集めて関心下の項目を出題し,得られた反応データから項目特性を推定する「事前テスト」が一般に行われる.
% しかし,事前テストの実施には多大なコストがかかる上に項目内容の漏洩リスクも生じる.

% この問題を解決するために,自然言語処理技術を用いて問題文から項目特性を推定する方法論が研究されている.
% 問題文を入力として項目特性値を出力する機械学習モデルを教師あり学習の枠組みで構築する手法~\cite{benedetto2023quantitative}や,質問応答システムを応用して人間受験者の項目反応を模倣する「仮想受験者」と呼ぶ技術~\cite{tomikawa2024adaptive,uto2024question,benedetto-etal-2024-using}などが提案されているが,いずれも訓練データとして大量の項目反応データが必要な課題が残っている.
% 他方,Benedetto et al.\cite{benedetto-etal-2024-using}は,LLMへのプロンプトとして,問題文と共に想定する受験者の特徴を与えることで,ゼロショットで受験者反応を模倣する仮想受験者モデルを提案している.しかし,この手法では,プロンプトに与える受験者の特徴が少数段階の能力レベルの指示のみに留まっており,細やかな知識状態の違いを模倣できていない.さらに,LLMによっては,適切に受験者レベルの調整ができない課題がある.これらはいずれも,プロンプトで与える受験者のプロフィールが荒いことに起因すると考えられる.

% これらの課題を踏まえ,本研究では,個別の受験者が有すると想定される知識・技能を受験者プロフィールとして明示的に定義し,それをプロンプトに組み込むことで,受験者反応模倣精度の改善を試みた.
% 具体的には,英語読解テストを対象として,ヨーロッパ言語共通参照枠(Common European Framework of Reference for Languages: CEFR) に基づくA1からC2までの各レベルの受験者プロフィールを, Can-Do Statement, Can't-Do Statement, および語彙レベル制約を用いて設計し,それらを制約条件としてLLMに与える枠組みを構築した.
% 提案手法の評価のため,指定した受験者レベルにおける項目困難度変化への応答傾向を評価するプロフィール感度(PS)と,項目難易度を固定した際の受験者レベル差に対する弁別性を評価する項目弁別力(QDP)の2指標を定義し定量的評価を行った.
% 実験の結果,Can-Do Statementと語彙制約を付与した条件では PS・QDPともに改善がみられ,複数制約の併用により一定の精度向上が確認された一方で,すべての制約を同時に適用した場合にはPS・QDPが低下し,制約の厳密な併用が必ずしも有効でないことも明らかになった.以上より,受験者プロフィールの精緻化による反応模倣精度の局所的改善は確認されたものの,受験者レベルや項目難易度に応じた正答率分布を安定的かつ一貫して再現する点には構造的な限界があることが示唆された.

個別適応的な学習支援やCBTにおいては,難易度や識別力などの項目特性が既知のテスト項目が重要となる.
しかし,項目特性を推定するためには人間受験者による事前テストが必要であり,多大なコストや問題漏洩のリスクが伴う.
この課題に対し,自然言語処理技術を用いた項目特性推定手法が提案されているが~\cite{tomikawa2024adaptive,uto2024question},この方法でも依然としてモデルの訓練に項目反応データを必要とする点が課題となっている.
他方で,近年では大規模言語モデル(LLM)を用いて仮想的な受験者を構築し,項目反応データを生成することで項目特性を推定する手法が注目されている\cite{benedetto-etal-2024-using}.
しかし,従来手法ではLLMに指定する受験者プロフィールが粗く,段階的な知識状態や技能差を十分に反映できていないという問題がある.

そこで本研究では,受験者が有すると想定される知識・技能を明示的に記述した受験者プロフィールをLLMへのプロンプトとして与えることで,仮想受験者の回答挙動がどの程度制御・再現可能か検討する.
具体的には,英語読解力テストを対象とし,ヨーロッパ言語共通参照枠(CEFR)に基づく Can-Do Statement および Can't-Do Statement,ならびに English Vocabulary Profile(EVP)に基づく語彙レベル制限を個別または組み合わせて付与し,その影響を比較した.
評価指標としては,正答率の項目難易度方向の単調減少を評価するプロフィール感度(PS)と,受験者プロフィール方向の単調増加を評価する項目弁別力(QDP)を用いた.
英文読解の多肢選択問題データセットを用いて実験を行い,各条件における正答率の変化を分析した.

実験の結果,一部の条件ではPSおよびQDPの局所的な改善が確認されたものの,全体としては正答率が高い水準に留まり,受験者レベルや項目難易度に応じた段階的な正答率の変化は十分に再現されなかった.
このことから,受験者プロフィールの精緻化によって特定の応答特性を局所的に強調することは可能である一方,Zero-shot 条件におけるLLM仮想受験者には,受験者反応を包括的に模倣する上での限界が存在することが示唆された.

\section{関連研究}

Benedetto et al.\cite{benedetto-etal-2024-using}は,LLMに特定の受験者レベルを指定して多肢選択問題への回答を生成させる,ゼロショット型の仮想受験者手法を提案した.
この手法では,受験者レベルの上昇に伴う正答率の単調増加性と,
同一レベルにおける項目難易度依存性の二つの観点から性能評価を行っている.
実験の結果,GPT-3.5を用いた場合には単調増加性が確認された一方で,
他のLLMではこの性質が十分に再現されないことが報告されている.
この要因として,受験者能力が「レベル1から5」といった数値ラベルのみで与えられており,
各レベルに対応する知識状態や読解行動が明示されていない点が指摘されている.
以上より,従来手法は少量の情報から受験者反応を生成できる利点を持つ一方で,
受験者プロフィールの表現が粗く,モデルによっては意図した項目反応を安定して再現できないという課題を有する.

\section{提案手法}

% 前述の課題を踏まえ,本研究では,受験者プロフィールを精緻化した際のLLM仮想受験者の回答挙動の変化を分析する枠組みを検討する.
% Listing~\ref{tab:prompt-template}で示した共通のプロンプトテンプレートに,英文読解問題の問題文・設問・回答選択肢とともに,想定する受験者のCEFRに基づく受験者レベルと,その受験者が有すると想定される知識・技能をまとめ整理した受験者プロフィールとともにLLMに与える.受験者プロフィールにはCEFR準拠の行動記述であるCan-Do Descriptorsおよび語彙リストであるEnglish Vocabulary Profileから作成した,表~\ref{tab:profile_constraints}で示したような指定した受験者レベルに対応した受験者の可能な行動を表すCan-Do Statement,不可能な行動を表すCan't-Do Statement,使用可能語彙範囲をlemma数で定量的に表した語彙レベル制約を組み合わせて付与し,受験者レベル・項目難易度ごとの正答率変化から受験者反応の模倣精度を評価する.

前述の課題を踏まえ,本研究では,個別の受験者が有すると想定される知識・技能を受験者プロフィールとして明示的に記述し,その情報をLLMにプロンプトとして与えることで,仮想受験者の回答挙動にどのような変化が生じるか分析する枠組みを検討した.具体的には,英語の読解力テストを対象とし,CEFRに基づく行動記述であるCan-Do Descriptorsおよび語彙リストであるEnglish Vocabulary ProfileからA1からC2の各スキルレベルに対応する受験者プロフィールを設計する.受験者プロフィールには,想定する受験者のCEFRレベルを指定する受験者レベルと,表~\ref{tab:profile_constraints}に挙げたような,指定した受験者レベルに対応した受験者の可能な行動を表すCan-Do Statement,指定した受験者レベルより上位の行動記述から作成した不可能な行動を表すCan't-Do Statement,使用可能な語彙範囲をlemma数で定量的に表した語彙レベル制約を組み合わせて構築する.受験者プロフィールはListing~\ref{tab:prompt-template}で示した共通のテンプレートのプレースホルダとして回答指示,問題文・設問・回答選択肢,出力形式とともに埋め込むことで,条件間でプロンプト構造の差異による影響を排除し,制約設計そのものの効果を比較可能とする.

\begin{lstlisting}[
    caption={共通プロンプトテンプレート},
    label={tab:prompt-template},
    basicstyle=\ttfamily\small,
    breaklines=true,
    lineskip=-1pt
]
You are simulating a language learner taking a reading comprehension test.
Your language level is CEFR {student_level}.
{profile_description}
Read the following passage and answer the question.
Passage:
{passage}
Question:
{question}
Choices:
{choices}
Select the best answer.
\end{lstlisting}

\begin{table*}[t]
\centering
\caption{CEFR B1 受験者プロフィールに含めた制約要素の概要}
\label{tab:profile_constraints}
\setlength{\tabcolsep}{6pt} % 列間余白をやや縮小
\begin{tabular}{p{3.2cm} p{7.2cm} p{6.8cm}}
\hline
制約名 & 説明 & 具体例(CEFR B1) \\
\hline
Can-Do Statement &
CEFR B1 レベルの受験者において「できる」と想定される読解行動を,
CEFR Can-Do Descriptors に基づいて明示的に記述する制約.
&
- Can understand the main points of clearly written,
straightforward texts and follow the plot of clearly structured
stories if the language is not too difficult.
\\
\hline
Can't-Do Statement &
CEFR B1 より上位レベルの Can-Do Descriptors を基に,
B1 受験者では実行できない読解行動を明示する制約.
&
- Cannot infer implicit meanings or nuanced opinions, nor fully
understand abstract or structurally complex texts.
\\
\hline
語彙制約 &
English Vocabulary Profile(EVP)に基づき,
指定したレベルの受験者が使用可能な語彙範囲を lemma 単位で
定量的に制限する制約.
&
- Use only EVP words up to CEFR B1 (A1--B1 words);
the total usable vocabulary size should not exceed approximately
2{,}000 lemmas.
\\
\hline
\end{tabular}
\end{table*}

\section{実験}

本実験では,CEFRに基づくB1からC2の4段階の難易度ラベルが付与された英語読解問題データセットであるCUP\&Aを用いた.また回答を生成させるLLMとしてGPT-4o-miniを使用した.プロンプトに与える受験者プロフィールとして,受験者レベルのラベルのみを与えるベースライン条件(baseline)に加え,Can-Do Statement,Can't-Do Statement,語彙制約を単独または組み合わせて付与した計8条件を設定した.

実験手順は次の通りである.まずはじめに,問題データを1パッセージ1設問とし,統一的に扱える形式へ変換した.
次に,条件に応じて含める制約を切り替えた仮想受験者の受験者プロフィールを組み立て,各設問ごとに,回答指示,問題文・設問・回答選択肢,出力形式とともにプロンプトテンプレートに埋め込む形でプロンプトを構築し,LLMに回答を生成させた.
さらに,得られた回答とデータセットに付属する正解ラベルを比較することで,正誤を判定し,受験者レベルと項目困難度の組み合わせごとに正答率を算出した.
最後に,人間受験者反応模倣精度を難易度方向の正答率変化,受験者レベル方向の正答率変化で評価するために,同一受験者レベルにおける項目難易度間の正答率差を総和したプロフィール感度(PS)と同一項目難易度における受験者レベル間の正答率差を総和した項目弁別力(QDP)の2つの指標を算出し,各条件間で評価指標を比較し,受験者プロフィール設計の影響を分析した.

\begin{table}[t]]
  \centering
  \caption{各実験条件におけるプロフィール感度(PS)および項目弁別力(QDP)}
  \label{tab:main_results}
  \begin{tabular}{lccc rr}
    \hline
    条件   & PS & QDP \\
    \hline
    Baseline   & 7.47 & 1.06 \\
    Prop w/ Can-Do     & 8.87 & 0.50 \\
    Prop w/ Can't-Do  & 7.84 & 0.50 \\
    Prop w/ Vocab  & 7.84 & 1.00 \\
    Prop w/ Can-Do \& Can't-Do   & 7.84 & 1.06 \\
    Prop w/ Can-Do \& Vocab  & 9.22 & 2.03 \\
    Prop w/ Can't-Do \& Vocab &  7.84 & 2.03 \\
    Prop w/ Can-Do \& Can't-Do \& Vocab & 8.53 & 0.00 \\
    \hline
  \end{tabular}
\end{table}

実験結果を表\ref{tab:main_results}に示す.
Can-Do Statementと語彙制約を付与した条件(Prop w/ Can-Do \& Vocab)でPSおよびQDPが最も高い値を示した. Prop w/ Can-Do \& Vocab 条件における,受験者レベルごとの項目難易度別正答率を図~\ref{fig:exp033_ps}に,項目難易度ごとの受験者レベル別正答率を図~\ref{fig:exp033_qdp}に示す.
項目難易度方向,受験者レベル方向の正答率変化の両方において,局所的な改善がみられるものの,全体としては正答率が高い水準に留まり,受験者レベルや項目難易度に応じた段階的な正答率の変化は十分に再現されなかった.

\begin{figure}[h]
    \centering
    \includegraphics[width=\linewidth]{img/exp033/difficulty_for_a1_c2_students_for_paper.eps}
    \caption{Prop w/ Can-Do \& Vocab における受験者プロフィールごとの項目難易度別正答率}
    \label{fig:exp033_ps}
\end{figure}

\begin{figure}[h]
    \centering
    \includegraphics[width=\linewidth]{img/exp033/student_level_for_b1_c2_questions_for_paper.eps}
    \caption{Prop w/ Can-Do \& Vocab における項目難易度ごとの受験者プロフィール別正答率}
    \label{fig:exp033_qdp}
\end{figure}

\section{まとめ}

本研究では,LLM を用いた仮想受験者モデルにおいて CEFR(A1--C2)に基づく受験者プロフィールをプロンプトとして与え,受験者レベルおよび問題難易度に応じた受験者反応の模倣可能性を検討した.
CEFR レベルのみを与えるベースラインに加え,Can-Do Descriptors や語彙レベル制限を組み合わせた複数条件を設定し,PS および QDP により回答挙動を評価した結果,一部条件で局所的な改善は確認されたものの,全体としては人間受験者に見られる正答率の段階的変化を安定的に再現するには至らなかった.
特に正答率が高止まりする傾向や,プロフィール制約の精緻化に伴う応答安定性の低下が観測され,本研究のアプローチは Zero-shot で受験者反応を生成できる利点を有する一方で,受験者レベルや項目難易度に応じた反応分布を精密に制御する点には構造的な限界があると結論付けられる.

{\small
\begin{thebibliography}{99}
% \bibitem{devlin2018bert} J. Devlin, M.-W. Chang, K. Lee, K. Toutanova, ``BERT: Pre-training of Deep Bidirectional Transformers for Language Understanding", arXiv preprint arXiv:1810.04805, 2018.
% \bibitem{benedetto2023quantitative} L. Benedetto, ``A quantitative study of NLP approaches to question difficulty estimation", Proceedings of the International Conference on Artificial Intelligence in Education, pp.428--434, 2023.
\bibitem{tomikawa2024adaptive} Y. Tomikawa, A. Suzuki, M. Uto, ``Adaptive Question–Answer Generation With Difficulty Control Using Item Response Theory and Pretrained Transformer Models", IEEE Transactions on Learning Technologies, Vol.17, pp.2186--2198, 2024.

\bibitem{uto2024question} M. Uto, Y. Tomikawa, A. Suzuki, ``Question Difficulty Prediction Based on Virtual Test-Takers and Item Response Theory", AIED'24: Workshop on Automatic Evaluation of Learning and Assessment Content, pp.1--11, 2024.
\bibitem{benedetto-etal-2024-using} L. Benedetto, G. Aradelli, A. Donvito, A. Lucchetti, A. Cappelli, P. Buttery, ``Using LLMs to simulate students' responses to exam questions", Findings of the Association for Computational Linguistics: EMNLP 2024, pp.11351--11368, 2024.
\end{thebibliography}
}
\end{document}
